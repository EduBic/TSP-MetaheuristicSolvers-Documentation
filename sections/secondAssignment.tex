\section{Second Assignment}

\subsection{Program}
	In order to resolve the ATSP with meta heuristics, I chose to implement the simple \textbf{Local Search} (LC) and the \textbf{Tabu Search} (TS).	From the literature we can see that either LS and mostly TS could be change in every details, starting solution, how to choice the neighbor and so on. For these reason I code the program in a modular way.
	
	\paragraph{SolverExecutor} For an easier testing I write a class \verb|SolverExecutor| in which we can add \verb|Solver|s with \verb|addSolver()| and execute sequentially with \verb|execute()| from the starting solution previously set with \verb|initRandomSolution()|. With this class is easy to modify the \verb|Main.cpp| file in order to set the environment of test in a pretty readable way and in a few statements.
	
	\paragraph{Solver} The other main parts of program are the \verb|Solver|s interface implemented by \verb|LocalSearchSolver| and \verb|TabuSearchSolver| classes. This permits to the developer to use solvers with different features into the \verb|SolverExecutor| class.
	
	\paragraph{LocalSearchSolver} can be built with the feature:
	\begin{itemize}
		\item Choice of neighbor: Best improvement and First Improvement.
	\end{itemize}
	
	\paragraph{TabuSearchSolver} can be built the following features:
	\begin{itemize}
		\item Size of tabu list;
		\item Maximum number of iteration of the algorithm;
		\item Maximum seconds of execution;
		\item Choice of neighbor: Best Improvement or First Improvement;
		\item Aspiration Criteria, if a neighbor has better value of current best solution it is chosen either if it is a tabu.
	\end{itemize}

	\subsubsection{When a move is a Tabu}
		Initially I discard a move if it is equal to one into the tabu list.
		I've tried to add also the condition that if a move is a \textit{specular} move into the tabu list that move is discarded. A move is \textit{specular} to another one if its \verb|from| is equal to its \verb|to| index and its \verb|to| is equal to its \verb|from| index.
		
		After some tests I saw that discard a specular move is not a convenient choice. It reduces the best value obtained from Tabu Search solver with a small tenure. For this I decide to not use that condition during the test phase.
		
	\subsubsection{Extensions of methods}
		I decide to not extend the Local Search for keep its simplicity as the main feature. For the Tabu search instead I implemented the Aspiration Criteria, I didn't implement any other functionality for intensification and diversification because I think that in order to diversify we can simply start another Tabu Search from a different initial random solution.
	
\subsection{Calibration}
	For calibration of the Tabu Search algorithm I look only to the tabu list length parameter. Hence I evaluate the test by the result obtained in \textbf{30 seconds} (CPU time), and from now on every test will be done with 30 seconds timeout. In fact if we increase the size of tabu list we need more time for check if one Move is a tabu.
	
	The test ran with 8 different random start solution for the instance \verb|rnd100.dat|. From the table\ref{tab:TS-calibration} we can see that the Aspiration Criteria (AC) increase the value of the best solution found and that a tenure of \textbf{180} is the best choice.
	
	In the tables I highlighted the rows that contain the best average value for TS without and with the AC.
	
	\begin{table}
		\centering
		\begin{tabular} {l l r}
			\toprule
			\textbf{Tenure} & \textbf{Features} & \textbf{Avg. Value} \\
			\midrule
			 80 & -  & 381.148  \\
			 80 & AC & 381.291 \\
			\midrule
			 100 & - & 377.131 \\
			 100 & AC & 376.455 \\
			\midrule
			 120 & - & 373.831 \\
			 120 & AC & 374.791 \\
			\midrule
			 140 & - & 376.768 \\
			 140 & AC &371.299 \\
			\midrule
			 160 & - & 373.419 \\
			 160 & AC & 371.108 \\
			\midrule
			\rowcolor{LightGray}
			 180 & - & 371.668 \\
			 \rowcolor{LightGray}
			 180 & AC & 370.189 \\
			\midrule
			 200 & - & 372.133 \\
			 200 & AC & 370.313  \\
			\bottomrule
		\end{tabular}
		\caption{\label{tab:TS-calibration100}Calibration results TS BI without or with AC on instance rnd100}
	\end{table}

	\begin{table}
		\centering
		\begin{tabular} {l l r}
			\toprule
			\textbf{Tenure} & \textbf{Features} & \textbf{Avg. Value} \\
			\midrule
			\rowcolor{LightGray}
			80 & -  & 629.8  \\
			\rowcolor{LightGray}
			80 & AC & 629.8 \\
			\midrule
			100 & - & 630.2 \\
			\rowcolor{LightGray}
			100 & AC & 629.8 \\
			\midrule
			120 & - & 630.5 \\
			\rowcolor{LightGray}
			120 & AC & 629.8 \\
			\midrule
			140 & - & 631.9 \\
			140 & AC & 629.9 \\
			\midrule
			160 & - & 632.2 \\
			160 & AC & 630.1 \\
			\midrule
			180 & - & 631.7 \\
			180 & AC & 630.7 \\
			\midrule
			200 & - & 634.0 \\
			200 & AC & 630.1  \\
			\bottomrule
		\end{tabular}
		\caption{\label{tab:TS-calibration60}Calibration results of TS BI without and with AC on instance tsp60}
	\end{table}
	
	\begin{table}
		\centering
		\begin{tabular} {l l r}
			\toprule
			\textbf{Tenure} & \textbf{Features} & \textbf{Avg. Value} \\
			\midrule
			80 & -  & 350.98  \\
			80 & AC & 349.94 \\
			\midrule
			100 & - & 349.34 \\
			\rowcolor{LightGray}
			100 & AC & 348.11 \\
			\midrule
			\rowcolor{LightGray}
			120 & - & 348.42 \\
			120 & AC & 348.12 \\
			\midrule
			140 & - & 348.77 \\
			140 & AC & 348.14 \\
			\midrule
			160 & - & 349.32 \\
			\rowcolor{LightGray}
			160 & AC & 348.11 \\
			\midrule
			180 & - & 349.36 \\
			180 & AC & 348.23 \\
			\midrule
			200 & - & 350.13 \\
			200 & AC & 348.19  \\
			\bottomrule
		\end{tabular}
		\caption{\label{tab:TS-calibration80}Calibration results of TS BI without and with AC on instance rnd80}
	\end{table}
	



	