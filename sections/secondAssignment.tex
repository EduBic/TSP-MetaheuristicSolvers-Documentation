\section{Second Assignment}

\subsection{Program}
	In order to resolve the ATSP with meta heuristics, I chose to implement the simple \textbf{Local Search} (LC) and the \textbf{Tabu Search} (TS).	From the literature we can see that either LS and mostly TS could be change in every details, starting solution, how to choice the neighbor and so on. For these reason I code the program in a modular way.
	
	\paragraph{SolverExecutor} For an easier testing I write a class \verb|SolverExecutor| in which we can add \verb|Solver|s with \verb|addSolver()| and execute sequentially with \verb|execute()| from the starting solution previously set with \verb|initRandomSolution()|. With this class is easy to modify the \verb|Main.cpp| file in order to set the environment of test in a pretty readable way and in a few statements.
	
	\paragraph{Solver} The other main parts of program are the \verb|Solver|s interface implemented by \verb|LocalSearchSolver| and \verb|TabuSearchSolver| classes. This permits to the developer to use solvers with different features into the \verb|SolverExecutor| class.
	
	\paragraph{LocalSearchSolver} can be built with the feature:
	\begin{itemize}
		\item Choice of neighbor: Best improvement and First Improvement.
	\end{itemize}
	
	\paragraph{TabuSearchSolver} can be built the following features:
	\begin{itemize}
		\item Size of tabu list;
		\item Maximum number of iteration of the algorithm;
		\item Maximum seconds of execution;
		\item Choice of neighbor: Best Improvement or First Improvement;
		\item Aspiration Criteria, if a neighbor has better value of current best solution it is chosen either if it is a tabu.
	\end{itemize}

	\subsubsection{When a move is a Tabu}
		I discard a move if it is equal to one into the tabu list, that is all moves into a tabu of length \verb|n| can not be used in the next \verb|n| iterations.
		
		\paragraph{Aspiration Criteria} With the aspiration criteria (AC) a tabu move can be selected. This case happens only when the tabu move makes a new solution that have a value greater than the actual best value found during the search.
		
		
	\subsubsection{Extensions of methods}
		I decide to not extend the Local Search for keep its simplicity as the main feature. For the Tabu search instead I implemented the Aspiration Criteria, I didn't implement any other functionality for intensification and diversification because I think that in order to diversify we can simply start another Tabu Search from a different initial random solution.
	



	