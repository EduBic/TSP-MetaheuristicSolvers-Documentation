\section{Introduction}
	The aim of this project is to familiarize with different methods to solve an Asymmetric Travelling Salesman Problem.
	\begin{itemize}
		\item Exact method that gives the optimal solution. This is implemented with the CPLEX API.
		\item Meta heuristic methods: local search and tabu search.
	\end{itemize}

	\subsection{Problem}
		The combinatorial optimization problem was:
		\begin{quote}
				A company produces boards with holes used to build electric frames.  Boards are positioned over a machines and a drill moves over the board, stops at the desired positions and makes the holes.  Once a board is drilled, a new board is positioned and the process is iterated many times.  Given the position of the holes on the board, the company asks us to determine the hole sequence that minimizes the total drilling time, taking into account that the time needed for making an hole is the same and constant for all the holes.
		\end{quote}
	
		This problem can be modelled as an Asymmetric Travelling Salesman Problem. The salesman is represent by the drill and cities by the holes in the board.
	
	\subsection{Document structure}
		The first part of reports show how the exact method, the local search and the tabu search were been implemented with C++ code. In the second part describes the test done and it shows the results obtained. There are two test sections, one with instances originated from random generator and one with real instances generated from real gerber (the standard format for represent PBCs) files.